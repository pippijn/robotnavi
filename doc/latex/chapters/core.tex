\chapter{Der Kern}

\section{Einführung}

Im Rahmen des Labors für Autonome Systeme haben wir einen Robotino \cite{festo}
programmiert, der einen konfigurierbare Kurs abfährt und dabei Hindernisse
umfährt. Zur Programmierung haben wir die von FESTO gelieferte C++ API
verwendet. Die Aufgabe war ursprünglich so angedacht, dass der Roboter eine
Menge von Strecken und Kreisbögen bekommt und diese dann nacheinander abfährt.
Wir haben die Aufgabe aber so verstanden, dass wir die Bahn selbst erzeugen
müssen aus einer Menge von Punkten. Diese Erzeugung haben wir mit OpenCASCADE,
einer freien Bibliothek für CAD Applikationen, realisiert. Zusätzlich zu der
Erkennung und Umfahrung der Hindernisse, haben wir eine Hindernisvermeidung
implementiert, die bereits lange vor einer möglichen Kollision kleine
Korrekturen an der Fahrbahn vornimmt, die eine Kollision in erster Linie
verhindern. Zuletzt haben wir noch ansatzweise ein Text-to-Speech Modul
eingebunden, das kritische Meldungen wie ,,obstacle in the way, rerouting'' zu
Sprache macht.

In diesem Kapitel wird die Hauptschleife (Main Loop) des Roboters erläutert.
Diese verwendet die beiden Komponenten ,,Kollisionsvermeidung'' und
,,Navigator'', welche in den beiden anderen Kapiteln weiter behandelt werden.


\section{Initialisierung}

Bei der Initialisierung der Steuersoftware wird zunächst ein Thread erzeugt,
welcher für die Sprache zuständig ist. Es werden dann vier Protokolle geöffnet,
welche jeweils einen Aspekt des Verlaufs festhalten:

\begin{itemize}
	\item \verb|travel.log|
		protokolliert die momentane Position.
	\item \verb|command.log|
		protokolliert den aktuellen absoluten Richtungsvektor, der von Navigator
		erhalten wurde.
	\item \verb|speed.log|
		protokolliert die aktuelle absolute Geschwindigkeit ohne Rücksicht auf
		Richtung.
	\item \verb|sensor.log|
		protokolliert die Eingangswerte der neun Sensoren an Bord des Robotinos.
\end{itemize}

Anschließend wird der erste Navigator initialisiert mit den Punkten, die
abgefahren werden sollen. Diese Punkte sind absolut anzugeben und bezeichnen
Punkte in einem millimetergenauen Koordinatensystem, welches von unserer
Steuersoftware gepflegt wird. Er wird zusätzlich der \verb|finished| Callback
im Navigator gesetzt, der bei der erfolgreichen Beendigung der Bahn aufgerufen
wird. Dieser neu erzeugte Navigator wird dann auf den Navigatorstack gelegt,
auf den später weiter eingegangen wird. Der Navigator wird mit einer
Schrittgenauigkeit \verb|navigator_step| initialisiert. Diese
Schrittgenauigkeit wird im Kapitel ,,Navigator'' genauer erläutert.

Dann werden in der Robotino-Software noch einige Callbacks gesetzt, die sich um
die Robotino-interne Fehlerbehandlung kümmern, worauf mit dem Verbindungsaufbau
begonnen wird. Ein Verbindungsversuch wird alle 20000 Millisekunden gestartet,
bis entweder ein Timeout eintritt oder der Verbindungsaufbau erfolgreich war.

Sobald die Verbindung steht, wird die Kamera, welche sich an Bord des Robotinos
befindet, initialisiert, damit sie höchstauflösende Bilder mit hoher
Kompressionsrate an den Client (unserer Steuersoftware) schickt. Diese Bilder
dienen lediglich der Visualisierung und werden nicht für die Programmlogik
benötigt. Das Bild der Kamera wird dann in einem OpenCV Fenster dargestellt.


\section{Main Loop}

Alle \verb|tick_time| Millisekunden wird eine Iteration des Main Loops
durchgeführt. Ein solcher Durchlauf sieht folgendermaßen aus, wobei auf die
Einzelheiten später tiefer eingegangen wird:

\begin{enumerate}
	\item Tastendruck einlesen
	
		Falls Tasten gedrückt wurden, werden die zugehörige Aktionen
		durchgeführt.

	\item Daten vom Robotino herunterladen

		Am Anfang der Berechnungskette werden die neuen Informationen des
		Robotinos in den \verb|RobotinoCom|, dem Kommunikator, der von der C++
		API zur Verfügung gestellt wird, geladen.

	\item Sensordaten aktualisieren

		Es werden ein mal pro Iteration die Sensordaten aktualisiert. Öfter würde
		sinnfrei sein, da die Daten nicht so schnell von den Sensoren kommen. Um
		die Aufrufkosten und Netzlast zu reduzieren, werden die Daten
		zwischengespeichert.

	\item Zielposition aktualisieren

		Anhand der gesammelten und bereits vorhandenen Informationen wird die
		neue absolute Zielposition berechnet und gesetzt.

	\item Geschwindigkeit setzen

		Damit die Zielposition erreicht wird, wird hier die Geschwindigkeit,
		sprich der Richtungsvektor, gesetzt.

	\item Bild anzeigen

		Falls die GUI aktiviert ist, wird nun das Kamerabild mit zusätzliche
		Informationen angereichert und angezeigt.

	\item Logs aktualisieren

		Zuletzt werden die momentane Position, Geschwindigkeit, etc. in die
		jeweiligen Protokolle geschrieben.
\end{enumerate}


\section{Die GUI}

Die graphische Benutzeroberfläche (GUI) zeigt nützliche Informationen über den
Zustand des Robotinos und der Bahn an. Es wird auch direkt über diese
Oberfläche die Möglichkeit geboten, den Robotino manuell zu kontrollieren. Dies
wird über den sogenannten ,,Override''-Modus erreicht. Solange der Robotino
sich in diesem Modus befindet, ist der Navigator ausgeschaltet und man kann ihn
frei positionieren.

\bild{screen1}{10cm}{Die Benutzeroberfläche}

Die Benutzeroberfläche zeigt ein Neuneck, an dessen Ecken die jeweiligen
Sensordaten zu sehen sind. Diese werden bei jeder Iteration aktualisiert. Die
Zahlen färben sich orange, falls die Sensorwerte einer Warndistanz
unterschreiten und Rot, falls sie der Panikdistanz unterschreiten. Ein rotes
Neuneck bedeutet, dass sich der Robotino im ,,Override''-Modus befindet. Blau
weist darauf hin, dass er momentan autonom handelt.

Dazu gibt es zwei Statuszeilen. Die obere Zeile zeigt folgende Informationen:

\begin{itemize}
	\item \verb|t|
		Die Zeit in Millisekunden, die seit der letzten Iteration vergangen ist.
		So war zum Beispiel in unseren Tests die \verb|tick_time| 10ms also
		dauerte eine Berechnung inklusive Netzwerkübertragung und Bildanzeige in
		der letzten Iteration auf Abbildung 1.1 3ms.

	\item \verb|v|
		Die momentane Geschwindigkeit in $x$- und $y$-Komponenten aufgeteilt.
		Diese Zahl kann negativ sein, was bedeutet, dass sich der Robotino
		rückwärts bewegt.

	\item \verb|req_v|
		Die gewünschte Geschwindigkeit. Unsere Software kümmert sich um einen
		gleichmäßigen, ruhigen Übergang zwischen Geschwindigkeiten. Bei jeder
		Iteration wird die tatsächliche Geschwindigkeit ,,langsam'' an die
		Zielgeschwindigkeit \verb|req_v| angepasst.

	\item \verb|avoid_v|
		Die von der Kollisionsvermeidung berechneten Geschwindigkeit, die jedes
		mal auf den momentanen Geschwindigkeitsvektor addiert wird.

	\item \verb|rot|
		Die momentane Rotationsgeschwindigkeit in °/s.
\end{itemize}

Die untere Statuszeile zeigt die momentane absolute Position in cm an.


\subsection{Bedienung im ,,Override''-Modus}

Der Robotino lässt sich manuell bedienen, wenn er sich im ,,Override''-Modus
befindet. Es stehen folgende Tastaturkommandos zur Verfügung:

\begin{itemize}
	\item \verb|q|

		Beendet die Main Loop, wodurch Schritte zur Terminierung der
		Netzwerkverbindung, dem Schließen der Protokolldateien und schlussendlich
		der Beendigung des Kontrollprogramms eingeleitet werden.

	\item \verb|a|

		Erhöht die Rotation nach rechts um 10°/s. Im Falle einer momentanen
		Linksrotation wird diese dadurch um 10°/s verringert.

	\item \verb|d|

		Erhöht die Rotation nach links um 10°/s. Im Falle einer momentanen
		Rechtsrotation wird diese dadurch um 10°/s verringert.

	\item \verb|w|

		Erhöht die \verb|req_v|, also die gewünschte Geschwindigkeit, in
		$x$-Richtung um \verb|override_step| mm/s.

	\item \verb|s|

		Verringert die \verb|req_v|, also die gewünschte Geschwindigkeit, in
		$x$-Richtung um \verb|override_step| mm/s.

	\item \verb|y|

		Erhöht die \verb|req_v|, also die gewünschte Geschwindigkeit, in
		$y$-Richtung um \verb|override_step| mm/s.

	\item \verb|x|

		Verringert die \verb|req_v|, also die gewünschte Geschwindigkeit, in
		$y$-Richtung um \verb|override_step| mm/s.

	\item \verb|z|

		Kehrt die gewünschte Geschwindigkeit um. Wenn sich also beispielsweise
		der Robotino momentan mit 200 mm/s in $x$- und 250 mm/s in $y$-Richtung
		bewegt, dann wird er nach dem Drücken auf \verb|z| anfangen, seine
		Geschwindigkeit zu verringern, um anschließend in die entgegengesetzte
		Richtung loszufahren.

	\item \verb|o|

		Aktiviert bzw. deaktiviert den ,,Override''-Modus.

	\item \verb|r|

		Setzt die Position und Geschwindigkeit ($x$/$x$/Rotation) zurück auf 0.

	\item \verb|e|

		Wie \verb|r|, behält jedoch die Position bei.

	\item \verb|1|

		Schaltet auf normale Ansicht der GUI.

	\item \verb|2|

		Zeigt nicht mehr die Daten und Bilder, wie in der normalen Ansicht, an,
		sondern nur noch die Konturen der Objekte. Diese Konturen werden von
		OpenCV erkannt. Ursprünglich dachten wir, es wäre unsere Aufgabe, diese
		Konturen zu verwenden, um einer auf dem Boden aufgeklebten Bahn zu
		folgen. Den Code dazu haben wir nicht verworfen.
\end{itemize}


\section{Bewegung}

\subsection{Kollisionsvermeidung}

In der Funktion zur Bestimmung der nächsten Zielposition, welche bei jeder
Iteration aufgerufen wird, wird als allererstes die Kollisionsvermeidung
aktiviert. Diese sorgt für direkte Verhinderung des schlimmsten Falles. Im
Falle einer Kollision würde der Robotino jede Orientierung verlieren, da er
komplett blind fährt, also nicht auf das Durchdrehen der Räder oder auf einer
Verschiebung durch einen Aufprall reagieren kann.


\subsection{Hindernisumfahrung und die Navigatoren}

Wie bereits erwähnt, besitzt die Steuersoftware einen \verb|stack|, also einen
Stapel mit Navigatoren. Dabei gibt es immer mindestens einen, nämlich den
Hauptnavigator, der den Robotino über die bei der Initialisierung konfigurierte
Bahn navigiert. Sollte nun ein Hindernis im Weg stehen, so gibt es zwei
Schritte, die nacheinander eingeleitet werden:

\begin{enumerate}
	\item Verringern der Geschwindigkeit auf 70\% bei Unterschreitung einer
		Warndistanz. Diese ist pro Sensor konfigurierbar über das Array
		\verb|warn_dist|. Standardmäßig ist die Warndistanz 18\% größer, als die
		Panikdistanz.
	\item Einleiten des Umfahrungsmechanismus bei Unterschreitung der
		Panikdistanz. Diese ist, ebenfalls pro Sensor, konfigurierbar über das
		Array \verb|panic_dist|.
\end{enumerate}

\subsubsection{Umfahrungsmechanismus}

Der Umfahrungsmechanismus benutzt einen weiteren Navigator, der den aktuellen
Navigator temporär ersetzt. Die Sequenz sieht folgendermaßen aus:

\begin{enumerate}
	\item Es werden nur eine bestimmte Anzahl Versuche gestartet, um einen
		bestimmten Punkt zu erreichen. Diese Anzahl ist über die Konstante
		\verb|max_retry| konfigurierbar. Falls diese Zahl überschritten wurde,
		werden alle Ausweichnavigatoren gelöscht, sodass nur noch der
		Hauptnavigator bleibt. Diesem wird dann mitgeteilt, dass der nächste
		Punkt auf der Strecke anvisiert werden soll.

	\item Ein neuer Navigator wird erzeugt, der folgende neue Koordinaten zum
		Abfahren bekommt:
		\begin{enumerate}
			\item Die aktuelle Position als Startposition
			\item Ein Meter entgegengesetzt der momentanen Fahrtrichtung von der
				aktuellen Position entfernt, hier Sicherheitsposition genannt.
			\item Ein Punkt, der auf der Mitte des Kreises zwischen neues Ziel und
				der Sicherheitsposition liegt. Es wird also ein Kreis gezogen,
				dessen Durchmesser der Abstand zwischen Zielposition und
				Sicherheitsposition ist. Der dritte Punkt ist auf 90° dieses
				Kreises.
			\item Das nächste Ziel des Navigators.
		\end{enumerate}

	\item Das Flag \verb|moving_back| wird gesetzt. Dieses Flag verhindert, dass
		der Ausweichmechanismus sofort in der nächsten Iteration wieder aktiviert
		wird, da der Robotino noch nicht auf die Sicherheitsposition gefahren
		ist. Solange dieser Flag gesetzt ist, wird nicht auf die Panikdistanz
		geachtet.

	\item Dem neuen Navigator wird als Handler für das Erreichen eins Punktes
		auf der Bahn eine Funktion mit dem Namen \verb|local_finished| übergeben,
		die den Ausweichnavigator entfernt, falls das Ende erreicht wurde. Falls
		die Sicherheitsposition erreicht wurde, wird das Flag \verb|moving_back|
		zurückgesetzt, damit der Umfahrungsmechanismus wieder aktiv ist.
\end{enumerate}

\bild{screen2}{10cm}{GUI im Falle einer Panikattacke}

Wie in Abbildung 1.2 erkennbar ist, wurde der Fuß, der im Weg stand, auf
Sensoren 0 und 8 erkannt (rot sagt: Panikdistanz wurde unterschritten). Kurz
vorher sieht man auf der Konsole, dass die Warndistanz ebenfalls unterschritten
worden war. Danach wird die Ausweichroute berechnet und auf der Konsole
ausgegeben. Unsere Software schreibt sehr viele Statusinformationen auf die
Konsole wie auch in einem Protokoll, damit später alle Bewegungen
nachvollziehbar sind. Die neue Routenberechnung geschieht auf einem Millimeter
genau, anders als die sehr genaue Darstellung in Abbildung 1.2 vermuten lässt.
Diese Zahlen werden mit \verb|nearbyint| gerundet.


\section{Geschwindigkeitsbestimmung}

Die Ansteuerung der Motoren muss gleitend passieren und darf nicht abrupt sein,
da sich sonst die Räder ohne tatsächlichen Effekt drehen, also unter dem still
stehenden Robotino durchdrehen. Dass sich der Robotino nicht bewegt, ist dabei
nicht das größte Problem. Das weitaus größere Problem ist, dass sich durch das
drehen der Räder ohne Vorwärtsbewegung der Robotino unberechenbar rotiert,
wodurch die Navigation unmöglich wird.

Aus diesem Grund haben wir die tatsächliche Geschwindigkeit von der
Zielgeschwindigkeit entkoppelt. Sie wird wie folgt berechnet:

Es seien gegeben:

\begin{itemize}
	\item \verb|requestx| und \verb|requesty|, die Zielgeschwindigkeiten in $x$-
		und $y$-Richtung.
	\item \verb|movex| und \verb|movey|, die tatsächliche Geschwindigkeitswerte,
		die an der Hardware übergeben werden.
	\item \verb|speed_step|, der Schritt, mit dem die Hardwaregeschwindigkeit in
		jeder Iteration an die Zielgeschwindigkeit angepasst wird.
	\item \verb|avoidx| und \verb|avoidy|, die Geschwindigkeiten, die von der
		Kollisionsvermeidung berechnet wurden.
\end{itemize}

Dann gilt für \verb|requestx|:

\[
	\text{movex} = \left\{ \begin{array}{rl}
		\text{requestx}							& \mbox{falls $|$requestx $-$ movex$| <$ speed\_step}	\\
		\text{movex} - \text{speed\_step}	& \mbox{falls movex $>$ requestx}							\\
		\text{movex} + \text{speed\_step}	& \mbox{falls movex $<$ requestx}							\\
	\end{array} \right.
\]

unter Beachtung der Regel, dass \verb|requestx| niemals größer werden kann als
1400 und niemals kleiner als -1400, da dies die maximal erreichbare
Hardwaregeschwindigkeit ist. Größere Werte würden die Odometrie, also die
Positionsbestimmung, welche auf Basis von Zeit und Geschwindigkeit
funktioniert, verfälschen.

Danach wird, falls die Warndistanz unterschritten wurde, die Geschwindigkeit
auf 75\% gedrosselt.

Zuletzt wird \verb|avoidx| auf die Geschwindigkeit addiert.

Für \verb|movey| wird analog verfahren.


\section{Sensoren}

Da die Sensoren ein hohes Maß an Rauschen und ungenauen Werten von sich geben,
speichern wir immer die letzten \verb|sensor_reads| Sensorwerte und bilden
daraus das arithmetische Mittel.


\section{Eine Beispielbahn und die Einstellungen dazu}

\subsection{Einstellungen}

In diesem Abschnitt werden die Werte für die in der vorangegangenen
Beschreibung benutzten Konstanten aufgelistet, welche für unseren spezifischen
Fall optimal waren. Sie wurden mit sehr hoher Sorgfalt ausgewählt und für einen
spezifischen Robotino eingestellt. Sie sind damit spezifisch für den Boden, den
Robotino, die Bahn, den Raum, die Temperatur und vielen weiteren Faktoren und
können damit nicht ohne Weiteres für andere Randbedingungen eingesetzt werden.

\begin{itemize}
	\item \verb|navigator_step|	= 100 mm
	\item \verb|max_speed|			= 200 mm/s
	\item \verb|max_avoid_speed|	= 150 mm/s
	\item \verb|override_step|		= 100 mm/s
	\item \verb|sensor_reads|		= 10
	\item \verb|tick_time|			= 10 ms
	\item \verb|speed_step|			= 50 mm/s/tick
	\item \verb|max_retry|			= 3 Versuche
	\item \verb|panic_dist|			=
	\begin{enumerate}
		\item 80
		\item 140
		\item 90
		\item 100
		\item 90
		\item 90
		\item 110
		\item 90
		\item 120
	\end{enumerate}
	\item \verb|warn_dist|			= jeweils 20 weniger als \verb|panic_dist|
\end{itemize}

Es ist zu Beachten, dass die Werte für die Distanzen unbekannte Maße haben und
durch Versuche ermittelt wurden. Ein genaues Maß zu ermitteln wäre aufgrund des
hohen Maßes an Sensorrauschen darüber hinaus sinnfrei gewesen.


\subsection{Die Beispielbahn}

Abbildung 1.3 zeigt die Bahn, wie sie vom Robotino gefahren wird, wenn keine
Hindernisse sich in den weg stellen. Abbildung 1.5 zeigt den Betrag der
tatsächlichen Hardwaregeschwindigkeit, wie sie auf der geplanten Bahn gesetzt
wird. Abbildung 1.4 stellt der geplanten Bahn eine Bahn gegenüber, die ein
Ausweichmannöver aufgrund eines Hindernisses beinhaltet. Abbildung 1.6 zeigt
die Geschwindigkeit auf dieser Bahn mit Ausweichmannöver.

Zuletzt haben wir noch einen Versuch mit linearer Bahn gestartet, bei dem man
sehr schön erkennt, wie ausgewichen wird.

\bild{bahn}{10cm}{Die geplante Bahn}
\bild{bahn_evade}{10cm}{Die geplante Bahn mit einem Ausweichmannöver}
\bild{bahn_speed}{8cm}{Geschwindigkeit auf der geplanten Bahn}
\bild{bahn_evade_speed}{8cm}{Geschwindigkeit auf der Bahn mit Ausweichmannöver}
\bild{linear_evade}{10cm}{Ein Ausweichmannöver auf einer linearen Bahn}


% vim:tw=79 sw=3 ts=3 noexpandtab
