\chapter{Navigator}

Die Aufgabe der Navigator Klasse ist es dem Robotino die für seine Navigation
erforderlichen Informationen zur Verfügung zu stellen. Damit dies überhaupt
möglich ist, müssen dem Navigator die Zielpunkte, die der Robotino abfahren
möchte, bekannt sein. Diese Zielpunkte werden dem Navigator mittels der Funktion
\verb|void initialise(vector<gp_Pnt> points)| übergeben.

Die übergebenen Punkte werden mittels des CAD-Kernel Open Cascade
(www.opencascade.org) zu einer B-Spline-Kurve interpoliert. Die interpolierte
Kurve entspricht der Strecke, die der Robotino abfahren soll um alle Zielpunkte
zu erreichen.


\section{Vektor-Navigation}

Nachdem der Navigator durch die Übergabe der Zielpunkte initialisiert wurde und
somit die abzufahrende Strecke ermittelt wurde, kann der Robotino durch Aufruf
der Funktion \verb|MathVector getDestination(gp_Pnt position)| den
Richtungsvektor in Kurvenrichtung, abhängig von seiner aktuellen Position,
abfragen. Die Funktion erwartet dabei als Argument einen Punkt der die aktuelle
Position des Robotino beinhaltet. Dieser Punkt wird orthogonal auf die intern
gehaltene Kurve durch die Zielpunkte projiziert. Von dem projizierten Punkt
ausgehend wird ein Folgepunkt in Kurvenrichtung ermittelt. Der Abstand zwischen
dem projizierten Punkt und dem Folgepunkt ist frei wählbar und wird dem Navigator
bei der Initialisierung übergeben. Dabei ist zu beachten, dass es sich bei
diesem Abstand nicht um einen Streckenabstand in Form einer festen Kurvenlänge
handelt, sondern um eine Erhöhung der der B-Spline-Kurve zu Grunde liegenden
nicht linearen U-Unterteilung.

Nachdem der projizierte Punkt und der Folgepunkt ermittelt wurden, wird ein
Vektor von der aktuellen Position des Robotino zu dem Folgepunkt gebildet und
zurückgeliefert. Anhand von diesem Vektor weiß der Robotino nun, in Abhängigkeit
seiner aktuellen Position, in welche Richtung er seine Fahrt fortsetzen muss um
die nächsten Zielpunkte zu erreichen. Abbildung 3.1 veranschaulicht diesen
Vorgang.

\bild{grafik_1}{10cm}{Richtungsbestimmung mit Vektor-Navigation}

Bei diesem Navigationsverfahren ist auch sichergestellt, dass der Roboter stets
der Kurve folgt. Denn wenn er aus irgendwelchen Gründen die Kurvenstrecke
verlässt wird aufgrund der Navigation in Abhängigkeit der aktuellen Position des
Roboters der zurückgelieferte Richtungsvektor den Roboter stets wieder auf die
korrekte Strecke zurückführen. Die einzigen Fehler die dabei auftreten können
sind Präzisionsfehler seitens des Robotino. Wenn z.B. die Räder des Robotino
Schlupf haben, hat dies zur Folge, dass die Position an der er sich zu befinden
glaubt nicht der realen Position entspricht. Dies würde dann zu einem Versatz
des Kurvenverlaufs in der realen Fahrstrecke führen.


Ein weiterer Vorteil der hier vorgestellten Vektor-Navigation ist, dass die
Vektorlänge je nach Kurvenabschnitt variiert. Durch die nicht lineare
Unterteilung der B-Spline-Kurve ist der Abstand zwischen dem projizierten Punkt
und dem Folgepunkt größer oder kleiner, was zu einem längeren oder kürzeren
Richtungsvektor führt. Wird nun der Richtungsvektor seitens des Robotino nicht
nur zur Ermittlung der nächsten Fahrtrichtung, sondern auch dessen Länge zur
Ermittlung der als nächstes zu fahrenden  Geschwindigkeit genutzt, fährt der
Roboter z.B. in Kurven langsamer als in geraden Abschnitten im Streckenverlauf.

Dieses Verhalten wird durch die interne Darstellung der B-Spline-Kurven
ermöglicht. Eine B-Spline-Kurve wird nicht linear in U-Werte von 0 bis N
unterteilt. Dabei werden kurvige Abschnitte in erheblich mehr Abschnitte
unterteilt als Abschnitte die keine Richtungsänderung beinhalten.

Abbildung 3.2 veranschaulicht diese Begebenheiten.

\bild{grafik-2-3}{10cm}{B-Spline-Kurve}


\section{Kreisbogen-Rotations-Navigation}

Bei diesem Navigationsverfahren wird die intern gehalten Kurve in gleichlange
Abschnitte unterteilt. Wie lang diese Abschnitte sein sollen kann durch den
Aufrufer in der dafür angelegten Methode \verb|getRotationSegments(double nmbSegs)|
bestimmt werden. Das dabei übergebene Argument \verb|nmbSegs| gibt die Anzahl
der Segmente in die die Kurve unterteilt werden soll an. Dabei ist zu beachten,
dass die Anzahl der Segmente nicht zu klein gewählt werden darf, da sonst die
Differenz zwischen dem tatsächlichen Segmentverlauf und dem für die Navigation
verwendeten Kreisbogens zu groß wird. Abbildung 3.3 zeigt ein Beispiel einer in
gleichlange Segmente unterteilten Kurve an.

\bild{grafik_4}{10cm}{Kurvensegmentierung}

Der Robotino erhält als Ergebnis von diesem Navigationsverfahren eine Folge von
Rotationsanweisungen mit der zugehörigen Richtung in die die Rotation erfolgen
soll, sowie die Information über die Segmentlänge der Segmente die die Kurve
unterteilen. Mit diesen Informationen ist es dem Robotino möglich die
vorgegebene Strecke abzufahren. Voraussetzung dafür ist, dass der Robotino bei
gleichbleibender Geschwindigkeit stets vorwärts fährt und innerhalb der Zeit, die
er zu dem Abfahren der stets gleichlangen Segmente benötigt, sich um den jeweils
angegebenen Winkel rotiert. Die Rotation muss natürlich je nach Kurvenverlauf
entweder nach links oder rechts erfolgen. Die Information über die Richtung in
die die Rotation erfolgen soll ist jeweils mit angegeben.

Der Winkel um den sich der Robotino im Verlauf der Fahrt entlang eines Segments
rotieren muss wird durch den Schnittwinkel zwischen den beiden Tangenten im
Anfangs- und Endpunkt des Kurvensegmentes bestimmt. Abbildung 3.4
veranschaulicht dies.

\bild{grafik_5}{10cm}{Rotationsbestimmung}

Da der Winkel allein nicht ausreicht um den Robotino die gewünschte Strecke
fahren zu lassen muss ermittelt werden, ob sich der Robotino auf seiner Fahrt
entlang des jeweiligen Kurvensegments nach links oder rechts rotieren muss.

Um dies zu ermitteln wird zunächst der Vektor zwischen dem Startpunkt und dem
Endpunkt des Segments ermittelt (siehe Abbildung 3.5).

\bild{grafik_6}{10cm}{Rotationsrichtungsbestimmung}

Anhand der Koordinaten des Vektors wird der Vektor in eine von acht
Richtungskategorien eingeordnet. Abbildung 3.6 zeigt die verschiedenen
Richtungskategorien auf.

\bild{grafik_7}{10cm}{Richtungskategorien}

Selbiges muss nun auch noch mit dem Vektor zwischen Segmentsstart- und
Segmentsendpunkt des vorherigen Segments getan werden.

Nach dem man nun den Richtungsvektor des aktuellen sowie des vorherigen
Segmentes hat, kann man anhand der Schnittwinkel der beiden Vektoren zur X-Achse
ermitteln ob der aktuelle Vektor rechts oder links von dem vorangegangen Vektor
zeigt.

Sind beispielsweise beide Vektoren von Typ 3 und ist der Schnittwinkel des
aktuellen Vektors zur X-Achse größer als der Schnittwinkel der X-Achse zum
vorangegangenen Vektor, so befindet sich der aktuelle Vektor rechts von dem
vorangegangen. Sollte der Winkel des aktuellen Vektors kleiner als der des
vorangegangen sein befindet sich der aktuelle links von dem vorangegangen
Vektor.

\bild{grafik_8}{10cm}{Vektoren im vierten Quadrant}

Die Überprüfung des Schnittwinkels zur X-Achse ist jedoch nur erforderlich wenn
die beiden Vektoren vom selben Typ sind. Sind sie nicht vom selben Typ, befinden
sich also nicht im selben Quadranten des Koordinatensystems, so kann man sofort
eine Aussage treffen. Ist beispielsweise der vorangegangene Vektor vom Typ 2 so
befinden sich alle Vektoren vom Typ 7, 8 und 1 links von dem Vorangegangenen,
alle Vektoren vom Typ 3, 4 oder 1 befinden sich rechts vom Vorangegangen. Sollte
der aktuelle Vektor vom Typ 6 sein ist es egal ob nach links oder rechts rotiert
wird.

\bild{grafik_9}{10cm}{Vektortypen im Einheitskreis}


\section{Vergleich der Navigationsverfahren}

Vorab muss an dieser Stelle erwähnt werden, dass wir nur die Vektor-Navigation
in der Praxis eingesetzt haben. Ein Praxistest mit der
Kreisbogen-Rotations-Navigation war uns in der vorgegebenen Zeit leider nicht
mehr möglich.

Auch wenn es uns nicht möglich war beide Verfahren in der Praxis zu testen, ist
davon auszugehen, dass beide Verfahren zweifellos in der Lage sind den Roboter
entlang der vorgegebenen Strecke an seine Zielpunkte zu navigieren. Die
Ungenauigkeiten die dabei auftreten können hängen bei beiden Verfahren nicht von
dem jeweiligen Algorithmus sondern von Umgebungseinflüssen wie z.B. der Schlupf
der Räder auf glatten Oberflächen.

Ein Vorteil der Vektornavigation ist, dass der Roboter auch von einer
inkorrekten Position stets wieder auf die korrekte Strecke zurückgeführt wird.
Dies könnte z.B. passieren, wenn der Roboter auf seiner Fahrt zu spät abbremst.
Da die die Vektor-Navigation stets von der aktuellen Position des Roboters
arbeitet wird der Richtungsvektor auch von einer Fehlposition ausgehend korrekt
ermittelt. Bei der Kreisbogen-Rotations-Navigation hingegen ist eine derartige
Korrektur nicht ohne weiteres möglich, denn es wird bei diesem Verfahren stets
davon ausgegangen das der Roboter mit einer konstanten Geschwindigkeit ein
Segment fehlerfrei abfährt. Eine Überprüfung seiner tatsächlichen Position
findet dabei nicht statt.

Ein klarer Nachteil der Vektor-Navigation ist jedoch, dass der Robotino nicht
mit der Kamera voraus entlang der Strecke fährt. Das wäre zwar möglich, würde
jedoch eine Ausrichtung vor jedem zu fahrenden Vektor erfordern, was wiederum zu
einer Verzögerung und somit zu einer nicht fließenden Fahrt führen würde. Bei
der Kreisbogen-Rotations-Navigation hingegen ist eine Fahrt mit der Kamera in
Fahrtrichtung problemlos gegeben.

Abschließen kann man wohl sagen, dass die Vektor-Navigation aufgrund ihrer
weniger aufwendigen Implementierung, für Einsätze bei denen eine Ausrichtung der
Kamera in Fahrtrichtung nicht erforderlich ist, zu empfehlen ist. Ist jedoch
eine Ausrichtung der Kamera erforderlich, ist die
Kreisbogen-Rotations-Navigation vermutlich die bessere Wahl.


% vim:tw=79 sw=3 ts=3 noexpandtab
