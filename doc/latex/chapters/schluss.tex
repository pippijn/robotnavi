\chapter{Erfahrungen}

In diesem kurzen Kapitel beschreiben wir unsere persönlichen Erfahrungen mit
der Robotino Hardware und der bereitgestellten C++ API, sowie allgemein mit dem
Labor und der Veranstaltung. Wir hatten alle noch nie etwas in dieser Richtung
getan. Das, was am nächsten dran war, war das Automatisierungslabor, bei dem
wir in STEP7 zwei Industrieroboter programmieren mussten. Diese Erfahrung war
in fast allen Punkten sehr anders als die Erfahrung im Autonome System Labor.


\section{Hardware}

Die Hardware des Robotinos war für unsere Verhältnisse sehr ungenau. Wir kommen
von der Softwarewelt, wo man mit beliebiger Genauigkeit rechnen und mit relativ
hoher Sicherheit sagen kann, dass die gleiche Eingaben immer das gleiche
Ergebnis liefern. Diese Erfahrung mit einer sich bewegenden, ungenauen
Hardware, war sehr interessant und lehrreich.

Wir haben viele verschiedene Methoden entwickelt, um gegen die Ungenauigkeit
vorzugehen, wie zum Beispiel eine konstante Rotation von 1°/s, die immer zu der
aktuellen Rotation addiert wird, um die Abweichung eines der Robotinos
entgegenzuwirken. Eine Ungenauigkeit, die erzeugt wird durch zu schnelles
hochfahren der Motoren existiert in der Form in der Softwarewelt auch nicht und
wir haben, um dafür eine gute Lösung zu finden, auch zwei Anläufe gebraucht.

Auch für die Sensoren mussten wir eine Lösung finden gegen das Rauschen. Das
Problem war, dass einige Sensoren eine so hohe Varianz hatten in ihren
Messwerten, dass die Panikdistanz entweder viel zu oft unterschritten wurde
oder, indem wir die Empfindlichkeit zu sehr heruntersetzten, nie unterschritten
wurde und der Robotino mit dem Hindernis kollidierte.


\section{Software}

Die C++ API war sehr einfach zu bedienen. Es war hinreichend dokumentiert und
hat jede Funktionalität geboten, die wir gebraucht haben. Zwar konnte die
verwendete Version keine interne Odometrie, aber vermutlich war unsere eigene
sowieso besser. Es wäre eine gute ,,zweite Meinung'' gewesen, die wir als
sekundäre Odometrie hätten verwenden können.


\section{Allgemein}

Das Labor wurde gut unterstützt, aber es wurde auch genügend Freiheit gelassen,
die Aufgabenstellung anders zu interpretieren. Besonders letzteres war für uns
das, was es auszeichnete gegenüber allen anderen Laborveranstaltungen, bei
denen man eine genaue Aufgabe bekommt und bei denen das Ergebnis genau mit den
Erwartungen übereinstimmen soll. Ein solches, striktes, Labor ist natürlich für
den Prüfer um vieles einfacher, weil dieser nur Punkte auf einer Liste
durchgehen und nachsehen muss, ob alles funktioniert.

Wir wünschen sowohl den unterstützenden Professoren als auch den zukünftigen
Laborteilnehmern ähnliche Erfahrungen, wie wir sie gemacht haben. Das Labor
sollte auf jeden Fall genau so weitergeführt werden, wie wir es erlebt haben.


% vim:tw=79 sw=3 ts=3 noexpandtab
